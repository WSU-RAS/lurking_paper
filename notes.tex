

\documentclass[11pt, conference, a4paper]{IEEEtran}
\usepackage[utf8]{inputenc}
\usepackage{amsmath}
\usepackage{amsfonts}
\usepackage{amssymb}
\usepackage{graphicx}
\usepackage[left=2cm,right=2cm,top=2cm,bottom=2cm]{geometry}
\author{D. Kelting, J. Maliauka, B. Manuel, C. Pereyda, A. Crandall and M. Schmitter-Edgecombe}
%Minimized names because they didn't fit.
%Daylan Kelting, Yulia Maliauka, Brittany Manuel, Christopher Pereyda, Aaron Crandall and Maureen Schmitter-Edgecombe


\title{RAS}


\begin{document}


\maketitle


\section{Main Message and Notes}
\begin{itemize}
	\item We are providing an effective way to select a location to place RAS while it is on standby. This location needs to be close enough to respond in a timely manner but unobtrusive to the resident.
	\item Use RAS -- Brief intro to him.
	\item Use past tense -- We, us, our.
\end{itemize}


\section{Title Ideas}
\begin{itemize}
	\item Robotics Placement in a Smart Home Environment
	\item RAS Placement in a Smart Home Environment 
	\item Lurking Robots While Avoiding Annoyance
	\item RAS Lurking in Unobtrusive Manner
	
\end{itemize}


\section{Abstract Outline}


\subsection{Motivation}
\begin{itemize}
    \item Home robotics are becoming more common
    \item As the ratio of elderly people to caretakers increases, having an 
        in-home nurse becomes impractical and unaffordable. We want to be able to place an assisted living robot 
        in a person's home to extend independent living.
    \item In order for future robotics to live in harmony with residents, assisted living robots need to stay close enough to be able to assist while also not being intrusive. 
\end{itemize}


\subsection{Problem Statement}
Where should RAS be during periods of non-use? This placement needs to be close enough to assist the resident while not being an annoyance. 


\subsection{Approach}
We collected resident location data using motion sensors embedded in a smart home environment. We derived the average location of the resident to estimate traffic patterns in the space. Using a variety of maps made from this data, we then assigned a score to every point in the space. The highest-scoring point within RAS' reach was determined to be the ideal location.


\subsection{Results}
In order to measure our algorithms success we conducted an informal study with human participants. After comparing the data gathered from our study to our algorithm's suggested point, we determined that we had provided good results. Our algorithm-generated point corresponded to either the first or second choice of our human participants.
\begin{itemize}
    \item \textless insert results from static study.\textgreater 
    \item \textless insert results from DTD evaluation.\textgreater
\end{itemize}



\subsection{Conclusions}
We present an algorithm for dynamic placement of a mobile home assistant robot 
when no tasks are required. Subjective analysis reveals our algorithm chooses 
good places to locate the robot. We hope this work provides a good base for 
future work in mobile robot placement in smart homes in the future.


\subsection{Abstract}

\subsubsection{Draft 1}

As the ratio of people in need of care to the number of caretakers increases, having an in-home nurse has become more impractical. Robotic assistance for the elderly is a promising solution to allow people to age in-place independently. One problem in robotic assistance is the location of the robot while it is idle. The robot should be close enough that it can effectively assist the person, but also not be intrusive. Here, we describe an algorithm that uses smart home data to choose places for the robot to sit, dynamically as people transition from place to place. 

\begin{itemize}
	\item \textless Insert results from all the things. \textgreater
\end{itemize}


\subsubsection{Draft 2}
\textless Brittany's Draft here \textgreater


\subsubsection{Draft 3}
\textless Third Draft here \textgreater


\section{Introduction}
Intro after the rest of the paper is written. 


\section{Related work -- Daylan}

\subsection{To Cite}
\begin{itemize}
    \item RAS Paper
    \item Smart home Paper(s)
    \item Nurse to Elderly Paper(s)
\end{itemize}


\section{Methods}


\subsection{Problem Overview}
In our problem scenario we are given a Simultaneous Location And Mapping (SLAM) map of a smart home and a real time stream of motion sensor data aligned with the SLAM map. Our task is to design an algorithm that will take this information and produce an optimal location for RAS to idle in. This location needs to be comprised of two key elements: ability to move to the resident in a timely manner and not being intrusive. 


\subsection{Solution Overview}
For our solution we took two sets of data, SLAM map and sensor data, then we built our own maps with the most relevant data to determine the optimal placement for RAS. 

The first data set we used to built a map was the SLAM map, from that we built a map representing all the areas that RAS can reach, this will be referred to as the reachability map. The reachability map was used to build two additional maps: cramped map and wall map. The cramped map is used to determine spaces that are cramped such as hallways, small offices or doorways. The wall map was made to push in, or waterfall, in walls so that RAS can gravitate toward places that are out of the way. 

The second data used was sensor data from the smart house. From the sensors we built two heatmaps: long term and short term. The long-term heatmap was used to determine the overall traffic through the smart house and build a path map. The short-term heatmap we took the weighted average of each point to find the center of recent activity.

From these maps we built we then combined them into one final map to referred to as the placement map. 


\subsection{Creating and Using the Heatmap}
The long-term heatmap used historcal data collected from the smart home to accumluate sensor triggers over time. This map is designed to capture long-term traffic patterns and locations that resident(s) spend most of their time. This new map is the path map which is the base for our placement map. 




The long-term heatmap uses historical data collected from the smarthome to sum up sensor triggers over time. This map is used to help the algorithm approximate long-term traffic patterns of the resident. Deciding where to place the charging base in the home can be a difficult process, since RAS needs to be able to assist the resident in a timely manner while also staying out of the way of everyday traffic. For this reason, the long-term heatmap is used to select the spot where the resident spends most of their time, and then additional information is used to select a spot that is out of the way - namely, the pathmap. 
The short-term heatmap, which is generated on a regular basis, assists the navigation algorithm continuously. When sensors in the smarthome are triggered, the short-term heatmap increments these sensors in the map, and routinely decays the heatmap by a factor of 50\% every 30 minutes. 


\subsection{Creating and Using the Weighted Average}
The weighted average class accepts a heatmap and generates a single point that represents the average location of the resident over a certain period of time. This point is calculated by adding up all of the values in the heatmap and the distances to them. This value is given a certain weight and used to generate a suggested charging base placement, or a realtime standby location.         


\subsection{Creating and Using the Path Map}
\textit{\textbf{tbd}}


\subsection{Creating and Using the Wall Map}
\textit{\textbf{tbd}}


\subsection{Creating and Using the Cramped Map}
\textit{\textbf{tbd}}

\subsection{Creating and Using the Reachability Map}
\textit{\textbf{tbd}} 

\subsection{The Lurking Algorithm}
The Lurking Algorithm uses all of the maps and points generated in the system to suggest a single point that RAS should move to at any given time. The different inputs are given different weights which have been tweaked multiple times to fine-tune the output value. The algorithm tends to centralize around the weighted average point, remain close to the walls detailed in the wall map, avoid tight areas illuminated by the cramped map, and cannot traverse into regions forbidden by the reachability map. By altering the weight, or importance, of each input, we can allow RAS to prioritize different regions. It may be beneficial to limit RAS to lounging in the center of the home, or only close to walls and furniture. Our final iteration of the algorithm attempts to balance all of these inputs in a way that maximizes availability to the resident and minimizes annoying the resident or getting stuck in tight spaces. 

\section{Results}


\section{Citations}
\begin{itemize}
    \item SLAM Paper
    \item RAS Paper
    \item Nurse to Eldery People Paper
    \item Other Paper
\end{itemize}



\end{document}
