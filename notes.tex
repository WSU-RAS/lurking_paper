\documentclass[11pt, conference, a4paper]{IEEEtran}
\usepackage[utf8]{inputenc}
\usepackage{amsmath}
\usepackage{amsfonts}
\usepackage{amssymb}
\usepackage{graphicx}
\usepackage[left=2cm,right=2cm,top=2cm,bottom=2cm]{geometry}
\author{D. Kelting, J. Maliauka, B. Manuel, C. Pereyda, A. Crandall and M. Schmitter-Edgecombe}
%Minimized names because they didn't fit.
%Daylan Kelting, Julia Maliauka, Brittany Manuel, Christopher Pereyda, Aaron Crandall and Maureen Schmitter-Edgecombe
\title{RAS}
\begin{document}
\maketitle


\section{Main Message and Notes}
\begin{itemize}
	\item We are providing an effective way to select a place for RAS to sit while on standby. This location needs to be close enough to respond in a timely manner and unobtrusive.
	\item Use RAS -- Brief intro to him.
	\item Use past tense -- We, us, our.
\end{itemize}


\section{Title Ideas}
\begin{itemize}
	\item Robotic Placement in a Smart Home Environment
	\item RAS Placement in a Smart Home Environment 
	\item Lurking Robots While Avoiding Annoyance
	\item RAS Lurking in Unobtrusive Manner
\end{itemize}


\section{Abstract Outline}


\subsection{Motivation}
\begin{itemize}
    \item Home robotics are becoming more common
    \item As the ratio of elderly people to caretakers increases, having an 
        in-home nurse becomes impractical and unaffordable. We want to be able to place an assisted living robot 
        in a person's home in an effort to extend independent living.
    \item In order for future robotics to live in harmony with residence, assisted living robots need 		to stay close enough to be able to assist while also not being intrusive. 
\end{itemize}


\subsection{Problem Statement}
Where shoud RAS be during periods of non-use? This placement needs to be close enough to assist the resident while not being an annoyance. 


\subsection{Approach}
We collected resident location data using motion sensors embedded in a smart home environment. We derived the average location of the resident to estimate traffic patterns in the space. Using a variety of maps made from this data, we then assigned a score to every point in the space. The highest-scoring point within RAS' reach was determined to be the ideal location.


\subsection{Results}
In order to measure our algorithms success we conducted an informal study with human participants. After comparing the data gathered from our study to our algorithm's suggested point, we determined that we had provided good results. Our algorithm-generated point corresponded to either the first or second choice of our human participants.
\begin{itemize}
    \item \textless insert results from static study.\textgreater 
    \item \textless insert results from DTD evaluation.\textgreater
\end{itemize}



\subsection{Conclusions}
We present an algorithm for dynamic placement of a mobile home assistant robot 
when no tasks are required. Subjective analysis reveals our algorithm chooses 
good places to locate the robot. We hope this work provides a good base for 
future work in mobile robot placement in smart homes in the future.


\subsection{Abstract Drafts}


\subsubsection{Draft 1}
As the ratio of people in need of care to the number of caretakers increases, having an in-home nurse has become more impractical. Robotic assistance for the elderly is a promising solution to allow people to age in-place independently. One problem in robotic assistance is the location of the robot while it is idle. The robot should be close enough that it can effectively assist the person, but also not be intrusive. Here, we describe an algorithm that uses smart home data to choose places for the robot to sit, dynamically as people transition from place to place. Insert results from all the things.


\subsubsection{Draft 2}
\textless Brittany's Draft here \textgreater


\subsubsection{Draft 3}
\textless Third Draft here \textgreater


\section{Introduction}
Intro after the rest of the paper is written. 


\section{Related work -- Daylan}
\subsection{To Cite}
\begin{itemize}
    \item RAS Paper
    \item Smart home Paper(s)
    \item Nurse to Elderly Paper(s)
\end{itemize}


\section{Methods}

\subsection{Problem Overview}
In our problem scenario we are given a Simultaneous Location And Mapping (SLAM) map of a smart home and a real time stream of motion sensor data aligned with the SLAM map. Our task is to design an algorithm that will take this information and produce an optimal location for RAS to idle in. This location needs to be comprised of two key elements: ability to move to the resident in a timely manner and not being intrusive. 


\subsection{Solution Overview}
For our solution we took two sets of data, SLAM map and sensor data, then we built our own maps with the most relevant data to determine the optimal placement for RAS. 

The first data set we used to built a map was the SLAM map, from that we built a map representing all the areas that RAS can reach, this will be referred to as the reachability map. The reachability map was used to build two additional maps: cramped map and wall map. The cramped map is used to determine spaces that are cramped such as hallways, small offices or doorways. The wall map was made to push in, or waterfall, in walls so that RAS can gravitate toward places that are out of the way. 

The second data used was sensor data from the smart house. From the sensors we built two heatmaps: long term and short term. The long-term heatmap was used to determine the overall traffic through the smart house and build a path map. The short-term heatmap we took the weighted average of each point to find the center of recent activity.

From these maps we built we then combined them into one final map to referred to as the placement map. 


\subsection{Using the SLAM Map}
Explain a bit more about the SLAM map, and the processing we did before we did the below maps.

\subsubsection{Reachability Map}
Daylan


\subsubsection{Wall Map}
The wall map was created by looking at each point within the SLAM map and "waterfalled" it out by a rate of UNKNOWN until we've reached a value of UNKNOWN or under. This waterfall process is best explained in the infographic below. As you can see with this the points are pushed out until we reach 10 or under. This process was repeated for every point in the SLAM map. We wanted this information so that RAS could gravitate toward walls and stay out of the residents way. 

This map was used in our final placement map by ADD/SUB its value by a rate of UNKNOWN.


\subsubsection{Cramped Map}
The cramped map is similar to the UNKNOWN map. The difference being that the KERNEL size is increaed to a 1.116 meter square. If any area was smaller than this square we would mark that area as cramped. We wanted this information so that RAS would be aware of cramped areas like nooks and hallways to GRAVIATE/AVOID those. 

This map was used in our final placement map by ADD/SUB its value by a rate of UNKNOWN.


\subsection{Using the Sensor Data}
Explain the sensor data, what kind there is and what specifically we used from the sensor data. I think we just used motion sensors. Double check that. 

\subsubsection{Long-Term Heatmap}
Our long-term heatmap uses historical data collected from the smart home environment. The specific set of data we focused on was the sensor triggers over time, this data set was used to determine approximate traffic patterns of the resident. We wanted this information to be able to have RAS idle near the resident.

This map was used in our final placement map by adding its value by a rate of UNKNOWN. t/f?


\textit{\textbf{Original}} The long-term heatmap uses historical data collected from the smarthome to sum up sensor triggers over time. This map is used to help the algorithm approximate long-term traffic patterns of the resident. Deciding where to place the charging base in the home can be a difficult process, since RAS needs to be able to assist the resident in a timely manner while also staying out of the way of everyday traffic. For this reason, the long-term heatmap is used to select the spot where the resident spends most of their time, and then additional information is used to select a spot that is out of the way - namely, the pathmap


\subsubsection{Path Map}
Daylan


\subsubsection{Short-Term Heatmap}
Similar to the long-term heatmap, the short-term heatmap uses sensor trigger data from the smart home. However, the short-term heatmap was constantly updated when a sensor was triggered. When these events would occure the map would update to have those areas have a heavier weight. We wanted this information to ensure(sp) that RAS would be near the resident as much as possible. Since we wanted this map to be current it had a decay rate of fifty percent every thirty minutes. 

This map was used in our final placement map by adding its value by a rate of UNKNOWN. t/f?


\textit{\textbf{Original}} The short-term heatmap, which is generated on a regular basis, assists the navigation algorithm continuously. When sensors in the smarthome are triggered, the short-term heatmap increments these sensors in the map, and routinely decays the heatmap by a factor of 50\% every 30 minutes. 


\subsubsection{Weight Map}
The weighted average class accepts a heatmap and generates a single point that represents the average location of the resident over a certain period of time. This point is calculated by adding up all of the values in the heatmap and the distances to them. This value is given a certain weight and used to generate a suggested charging base placement, or a realtime standby location.


\subsection{Putting it All Together}
The Lurking Algorithm uses all of the maps and points generated in the system to suggest a single point that Ras should move to at any given time. The different inputs are given different weights which have been tweaked multiple times to fine-tune the output value. The algorithm tends to centralize around the weighted average point, remain close to the walls detailed in the wall map, avoid tight areas illuminated by the cramped map, and cannot traverse into regions forbidden by the reachability map. By altering the weight, or importance, of each input, we can allow Ras to prioritize different regions. It may be beneficial to limit Ras to lounging in the center of the home, or only close to walls and furniture. Our final iteration of the algorithm attempts to balance all of these inputs in a way that maximizes availability to the resident and minimizes annoying the resident or getting stuck in tight spaces. 


\section{Discussion}
Limits of:
\begin{itemize}
    \item Results
    \item Time
    \item Prelim study, most people were researchers (grad, undergrad students)
    \item Not generalized
    \item only worked with one layout
\end{itemize}


\section{Future Works}
Connect it back to how you'd like to fix problems in discussion.


\section{Conclution}
Delay until finished.


\section{Acknowledgements}
\begin{itemize}
    \item Mentors: Crandall, Holder, Cook, Schmitter-Edgecombe, Chris?
    \item Money people
    \item Other?
\end{itemize}


\section{References}
\begin{itemize}
    \item SLAM Paper
    \item RAS Paper
    \item Nurse to Eldery People Paper
    \item Other Paper
\end{itemize}


\end{document}