

\documentclass[11pt, conference, a4paper]{IEEEtran}
\usepackage[utf8]{inputenc}
\usepackage{amsmath}
\usepackage{amsfonts}
\usepackage{amssymb}
\usepackage{graphicx}
\usepackage[left=2cm,right=2cm,top=2cm,bottom=2cm]{geometry}
\author{RAS Team}
\title{Paper Notes}


\begin{document}

\maketitle

\section{random notes}

We are providing a simple, effective, and accurate way to select unobtrusive 
places for a robot to sit in standby while staying close enough to the resident to be able to assist them in a timely manner.


\section{abstract}

\subsection{motivation}
why do we care about the problem?
\begin{itemize}
    \item home robotics are becoming more common
    \item as the ratio of elderly people to caretakers increases, having an 
        in-home nurse becomes impractical. We want to be able to place a robot 
        in a person's home to act as an in-home assistant so the person can continue to live 
        independently for as long as possible.
    \item in order for the robot to live in harmony with the residents, the robot needs 
        to stay close enough to be able to assist the person as 
        needed, but also out of the way.
\end{itemize}

\subsection{problem statement}
Where should the robot be when it is on standby so that it is able to assist people in a timely manner but not obstruct or annoy residents? 

\subsection{approach}
Using motion sensors embedded in the home we derive the average location of the 
residents to estimate traffic in the space. Using maps made from this data, we assign a score to every point in the space and return the highest-scoring point within the robot's reach. 

\subsection{results}
Points taken from the system that were evaluated subjectively provide good results since they corresponded with either the first or second choice of human participants.\textless insert results from static 
study\textgreater \textless insert results from DTD evaluation.\textgreater

\subsection{conclusions}
We present an algorithm for dynamic placement of a mobile home assistant robot 
when no tasks are required. Subjective analysis reveals our algorithm chooses 
good places to locate the robot. We hope this work provides a good base for 
future work in mobile robot placement in smarthomes in the future.

\subsection{drafts}
\subsubsection{draft 1}
As the ratio of people in need of care to the number of caretakers increases 
having an in-home nurse becomes more expensive and impractical. Robotic 
asistance of the elderly is a promising solution to allow people to age 
in-place more independently. A problem in robotic asistance is where to place 
the robot when no assistance is needed. The robot should be close enough that 
it can effectively assist the person if it should become needed, but also out 
of the way. Here, we describe an algorithm that uses smart home data to choose 
places for the robot to sit, dynamically as people transition from place to 
place. \textless insert results from all the things.\textgreater


\section{Introduction}

Probably do introduction after we finish the rest of the paper

\section{Related work}

\subsection{To Cite}
\begin{itemize}
    \item original RAS paper
    \item smart home papers
\end{itemize}

\section{Methods}

\subsection{Notes}
subsections to create:
\begin{itemize}
    \item re-introduce the problem?
\end{itemize}

\subsection{Problem}
We are given a SLAM (Simultaneous Location And Mapping) map of the smart home
and a real time stream of motion sensor data aligned to that slam map.
Our task is to find a place in the home that allows the robot to move to the
residents in as short a time as possible (Distance to delivery) while also
remaining out o fthe way should the residents need to move to a new area of
the home.

\subsection{Overview of the Solution}
Using the SLAM map we build a map representing all areas the robot can reach.
Using the sensor data we build two heatmaps, a long term heatmap and a short
term heatmap. From the long term heatmap we build a map estimating the traffic
through the house, which we call the path map. From the short term heatmap we 
take the weighted average of each point to find the center of recent resident 
activity. Both the weighted average point and the path map are then used to
score each point in the reachability map. We then return the maximum scoring
point as the point the robot should be.

\subsection{Creating and Using the Heatmaps}
The long-term heatmap uses historical data collected from the smarthome to sum up sensor triggers over time. This map is used to help the algorithm approximate long-term traffic patterns of the resident. Deciding where to place the charging base in the home can be a difficult process, since RAS needs to be able to assist the resident in a timely manner while also staying out of the way of everyday traffic. For this reason, the long-term heatmap is used to select the spot where the resident spends most of their time, and then additional information is used to select a spot that is out of the way - namely, the pathmap. 
The short-term heatmap, which is generated on a regular basis, assists the navigation algorithm continuously. When sensors in the smarthome are triggered, the short-term heatmap increments these sensors in the map, and routinely decays the heatmap by a factor of 50\% every 30 minutes. 

\subsection{Creating and Using the Reachability Map}
\textit{\textbf{tbd}} 

\subsection{Creating and Using the PathMap}
\textit{\textbf{tbd}}

\section{Citations}
\subsection{to cite}
\begin{itemize}
    \item SLAM paper
\end{itemize}



\end{document}
