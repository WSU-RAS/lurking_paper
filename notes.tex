

\documentclass[11pt, draft, a4paper]{IEEEtran}
\usepackage[utf8]{inputenc}
\usepackage{amsmath}
\usepackage{amsfonts}
\usepackage{amssymb}
\usepackage{graphicx}
\usepackage[left=2cm,right=2cm,top=2cm,bottom=2cm]{geometry}
\author{D. Kelting, J. Maliauka, B. Manuel, C. Pereyda, A. Crandall and M. Schmitter-Edgecombe}
%Minimized names because they didn't fit.
%Daylan Kelting, Julia Maliauka, Brittany Manuel, Christopher Pereyda, Aaron Crandall and Maureen Schmitter-Edgecombe
\title{RAS}
\begin{document}
\maketitle

\section{Main Message and Notes}
\begin{itemize}
        \item We are providing an effective way to select a place for RAS to sit while on standby. This location needs to be close enough to respond in a timely manner and unobtrusive.
        \item Use RAS -- Brief intro to him.
        \item Use past tense -- We, us, our.
\end{itemize}


\section{Title Ideas}
\begin{itemize}
        \item Robotic Placement in a Smart Home Environment
        \item RAS Placement in a Smart Home Environment 
        \item Lurking Robots While Avoiding Annoyance
        \item RAS Lurking in Unobtrusive Manner
        \item Dynamic Robot Placement in Smart Environments
\end{itemize}


\section{Abstract Outline}


\subsection{Motivation}
\begin{itemize}
    \item Home robotics are becoming more common
    \item As the ratio of elderly people to caretakers increases, having an 
        in-home nurse becomes impractical and not affordable. We want to be able to place an assisted living robot in a person's home in an effort to extend independent living.
    \item In order for future robotics to live in harmony with residence, assisted living robots need to stay close enough to be able to assist while also not being intrusive. 
\end{itemize}


\subsection{Problem Statement}
Where should RAS be during periods of non-use? This placement needs to be close enough to assist the resident while not being an annoyance. 


\subsection{Approach}
We collected resident location data using motion sensors embedded in a smart home environment. We derived the average location of the resident to estimate traffic patterns in the space. Using a variety of maps made from this data, we then assigned a score to every point in the space. The highest-scoring point within RAS' reach was determined to be the ideal location.


\subsection{Results}
In order to measure our algorithms success we conducted an informal study with human participants. After comparing the data gathered from our study to our algorithm's suggested point, we determined that we had provided good results. Our algorithm-generated point corresponded to either the first or second choice of our human participants.
\begin{itemize}
    \item \textless insert results from static study.\textgreater 
    \item \textless insert results from DTD evaluation.\textgreater
\end{itemize}


\subsection{Conclusions}
We present an algorithm for dynamic placement of a mobile home assistant robot 
when no tasks are required. Subjective analysis reveals our algorithm chooses 
good places to locate the robot. We hope this work provides a good base for 
future work in mobile robot placement in smart homes in the future.


\subsection{Abstract}
As the ratio of people in need of care to the number of caretakers increases, having an in-home nurse has become more impractical. Robotic assistance for the elderly is a promising solution to allow people to age in-place independently. One problem in robotic assistance is the location of the robot while it is idle. The robot should be close enough that it can effectively assist the person, but also not be intrusive. Here, we describe an algorithm that uses smart home data to choose places for the robot to sit, dynamically as people transition from place to place. In our approach we collected data using sensors in our smart home environment. We derived the average location of the resident to estimate traffic patterns in the space; additionally, we used a variety of maps made from this data to adjust for walls, tight spaces, traffic and areas the robot could reach. Once all of this information is complied, the highest-scoring point within reach was determined to be the ideal location. To validate our results we conducted an informal study. [INFO ABOUT STUDY]. Results are pending as our study finishes.

\end{document}
