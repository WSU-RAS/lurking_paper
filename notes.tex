

\documentclass[11pt, conference, a4paper]{IEEEtran}
\usepackage[utf8]{inputenc}
\usepackage{amsmath}
\usepackage{amsfonts}
\usepackage{amssymb}
\usepackage{graphicx}
\usepackage[left=2cm,right=2cm,top=2cm,bottom=2cm]{geometry}
\author{RAS Team}
\title{Paper Notes}


\begin{document}

\maketitle

\section{random notes}

We are providing a simple, effective, and accurate way to find unobtrusive 
places for a robot to sit while being close to people.


\section{abstract}

\subsection{motivation}
why do we care about the problem?
\begin{itemize}
    \item home robotics are becoming more common
    \item as the ratio of elderly people to caretakers increases having an 
        in-home nurse becomes impractical. We want to be able to place a robot 
        in a persons home to act as an in-home assistant so the person can stay 
        independent for longer.
    \item in order for the robot to live in harmony with people the robot needs 
        to stay close enough to be able to interact with the person when 
        needed, but also out of the way.
\end{itemize}

\subsection{problem statement}
Where should the robot be when it does not need to interact with any people.

\subsection{approach}
Using motion sensors embedded in the home we derive the average location of the 
residents and an estimate of traffic through the space. Using these two 
features we score every point in the space and output the highest scoring 
reachable point.

\subsection{results}
Points taken from the system and evaluated subjectively provide good results, 
being either the top 1 or 2 of human judged placements as compared to other 
places chosen randomly in the space. \textless insert results from static 
study\textgreater \textless insert results from DTD evaluation.\textgreater

\subsection{conclusions}
We present an algorithm for dynamic placement of a mobile home assistant robot 
when no tasks are required. Subjective analysis reveals our algorithm chooses 
good places to locate the robot. We hope this work provides a good base for 
future work in mobile robot placement in smarthomes in the future.

\subsection{drafts}
\subsubsection{draft 1}
As the ratio of people in need of care to the number of caretakers increases 
having an in-home nurse becomes more expensive and impractical. Robotic 
asistance of the elderly is a promising solution to allow people to age 
in-place more independently. A problem in robotic asistance is where to place 
the robot when no assistance is needed. The robot should be close enough that 
it can effectively assist the person if it should become needed, but also out 
of the way. Here, we describe an algorithm that uses smart home data to choose 
places for the robot to sit, dynamically as people transition from place to 
place. \textless insert results from all the things.\textgreater


\section{Introduction}




\section{Related work}



\end{document}
